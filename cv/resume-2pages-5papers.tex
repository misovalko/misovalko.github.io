\documentclass{resume}
\renewcommand{\categoryfont}{\sc}

\def\Cplusplus{{\rm C\raise.5ex\hbox{\small ++}}}

%
% set the space used for category titles here:
% use the same value for oddsidemargin and marginparwidth [the latter 
% 		will be reset to account for marginparsep]
% 
\setlength{\oddsidemargin}{1in}
\setlength{\marginparwidth}{1in}
% 
% calculate other dimensions [textwidth and evensidemargin] 
% in function of oddsidemargin and marginparwidth: 
% would be nicer to put in the class file...
%
\addtolength{\marginparwidth}{-\marginparsep}
 \setlength{\evensidemargin}{\oddsidemargin}
 \setlength{\textwidth}{\paperwidth}
 \addtolength{\textwidth}{-2in}
 \addtolength{\textwidth}{-2\oddsidemargin}
 \addtolength{\textwidth}{\marginparwidth}
 \addtolength{\textwidth}{\marginparsep}
%
%
\setlength{\topmargin}{-0.5in}
%
%
\renewcommand{\labelcitem}{$\diamond$}
\renewcommand{\labelitemi}{$\cdot$}
\newcommand{\first}{$1^{\mbox{\scriptsize st}}$\ }
\newcommand{\second}{$2^{\mbox{\scriptsize nd}}$\ }
\newcommand{\third}{$3^{\mbox{\scriptsize rd}}$\ }
\newcommand{\fourth}{$4^{\mbox{\scriptsize th}}$\ }
\newcommand{\fifth}{$5^{\mbox{\scriptsize th}}$\ }
\newcommand{\sixth}{$6^{\mbox{\scriptsize th}}$\ }
\newcommand{\nineth}{$9^{\mbox{\scriptsize th}}$\ }
\newcommand{\comment}[1]{}


\author{Michal Valko}
% ------ Address --------------------------------------------------------

\address{Stealth Startup\\
	San Francisco, California, US\\
	Paris, France\\		
	}{
	%67 Avenue du Peuple Belge, Appt 16\\
	%59800 Lille, France\\
%        Cell phone: \emph{please request}\\
	%Cell EU: +421 (908) 191217 \\
        % Cell USA: +1 (412) 499--3474\\
	%\mbox{\small\tt }\\
	\mbox{\small\tt http://researchers.lille.inria.fr/\~{}valko/}\\
	\mbox{\small\tt michal.valko\,@\,inria.fr}\\
	\mbox{\small\tt +33 3 59 57 7801}\\
}


\begin{document}
\maketitle

% \begin{category}{Objective}
% \citemnobullet 
%  Seeking a position in which I can continue doing research in \emph{machine learning}.
% \end{category}


% ------- Education ---------------------------------------------------
\begin{category}{Experience}
\citem{Inria -- team SequeL}, Lille, France \\
Experienced Junior Scientist - CR1 (2014 -- $\dots$)
\citem{ENS Cachan -- Master 2 MVA}, Cachan, France \\
External Lecturer - CEV (2014 -- $\dots$)
\citem{Inria -- team SequeL}, Lille, France \\
Junior Scientist - CR2 (2012 -- 2014)
\citem{Inria -- team SequeL}, Lille, France \\
Postdoctoral Researcher (2011 -- 2012), Advisor:  \emph{R\'emi Munos}
\citem{Intel Research},  Santa Clara, CA, USA\\
Research Intern (2009, 2010), Advisor:  \emph{Branislav Kveton}

\end{category}
\begin{category}{Education}
\citem{University of Pittsburgh}, Pittsburgh, PA \\ %(GPA 4.0) \\
PhD in Machine Learning, August 2011.\\
Thesis: \emph{Adaptive Graph-Based Algorithms}, Advisor: \emph{Milos Hauskrecht}
\citem{Comenius University Bratislava}, Slovakia\\% (GPA 3.96) \\
MSc.\,, Summa cum laude in Computer Science, June 2005.\\
Majors: Artificial Intelligence  and Mathematical Methods of CS\\ %(GPA 4.0) and
Thesis title: \emph{Evolving Neural Networks for Statistical Decision Theory.}
\end{category}
\begin{category}{Selected Awards}
	\citemnobullet Inria award for scientific excellence:
Prime d'excellence scientifique (2014 - 2017)
	\citemnobullet Compunetix Best Research Award at Computer Science 
Department (2008 
and 2011)	
	\citemnobullet University of Pittsburgh Honors Convocation 2009 
Recognition 
       \citemnobullet Andrew Mellon Predoctoral Fellowship (Fall 2008, Summer 
2009)	
\end{category}

% --------- Research ----------------------------------------------------

\begin{category}{Research Interests}
\citemnobullet machine learning, bandit theory, minimal feedback, 
online learning, sequential learning, graph-based methods, inverse 
reinforcement learning, semi-supervised learning
\end{category}


% -------- Publication --------------------------------------------
\begin{category}{Selected publications }
\citembullet
Jean-Bastien Grill, {\bf Michal Valko}, R\' emi Munos:
\emph{Black-box optimization of noisy functions with unknown smoothness},
Neural Information Processing Systems
({\sf NIPS 2015}) 


\citembullet
Tom\'a\v s Koc\' ak, Gergely Neu, {\bf Michal Valko}, R\' emi Munos: Efficient 
\emph{Learning by Implicit Exploration in Bandit Problems with Side 
Observations}, ({\sf NIPS 2014})

\citembullet
Alexandra Carpentier, {\bf Michal Valko}: \emph{Extreme Bandits},
({\sf NIPS 2014})

\citembullet
Gergely Neu, {\bf Michal Valko}: \emph{Online Combinatorial Optimization with 
Stochastic Decision Sets and Adversarial Losses},  ({\sf NIPS 2014})

\citembullet
{\bf Michal Valko}, R\' emi Munos, Branislav Kveton, Tom\'a\v s Koc\' ak:
\emph{Spectral Bandits for Smooth Graph Functions},
({\sf ICML 2014})  {\bf [oral presentation]}


\end{category}

\begin{category}{Service Activities}
\citembullet Elected member of Inria Evaluation Committee (CE Inria 2014 -- 2015, 2015 -- 2019)
\citembullet Organizing Committee: JFPDA (2013),  Research project reviewer: FNRS (2014 -- now)
\citembullet Program Committee: AAAI (2012, 2015), IJCAI (2015), RLDM (2015), EWRL 
(2012, 2015), JFPDA (2014)
\citembullet  Reviewer: NIPS (2012-2015), ICML (2012-2015), COLT
(2014), UAI (2011-2012), AISTATS (2016), IJCAI (2009), KDD (2011), AAAI (2009, 2014), ECML
(2012), MEDINFO (2010)
\citembullet INTEL/Inria - Algorithmic Determination of IoT Edge Analytic -
2013 (project leader)
\citembullet  European FP7 grant (CompLACS), ANR grant (ExtraLearn), NIH grants
\end{category}



%\begin{category}{Publications 
%under review}

%\citemnobullet  
%{\bf Michal Valko}, Gregory F.~Cooper, Branislav Kveton, Milos Hauskrecht:
%\emph{Fast Approximate Conditional Anomaly Detection with Random Walks}, 
%17th ACM SIGKDD Conference on Knowledge Discovery and Data Mining ({\sf KDD 2011}) 
%    (\emph{submitted})

%\citembullet
%Branislav Kveton, Kent Lyons, Lily Mummert, {\bf Michal Valko}
%\emph{Face-Based Authentication on Mobile Devices using Online Learning}, 
%Twenty-Fifth AAAI Conference on Artificial Intelligence ({\sf AAAI 2011}) 
%  (\emph{submitted})



%\citembullet
%  Wendy W. Chapman, John N. Dowling, Gregory F. Cooper, Milos Hauskrecht, {\bf Michal Valko}, Will
%  Bridewell: \emph{Identifying Acute Lower Respiratory Syndrome from Emergency Department Texts},
%  Journal of Biomedical Informatics ({\sf JBI 2011,} \emph{submitted})
%\end{category}


\begin{category}{Students}
\citembullet \textit{Daniele Calandriello}, 2014 -- 2016, PhD.\, student,
Inria, with A. Lazaric and R. Munos
\citembullet \textit{Jean-Bastien Grill}, 2014 -- 2016, PhD.\, student,
Inria, with R\' emi Munos
 \citembullet \textit{Tom\'a\v s Koc\' ak}, 2013 -- 2016, PhD.\, student,
Inria, with R\' emi Munos
\end{category}


%\newpage
\begin{category}{Invited Talks}
\citembullet
{\bf  Michal Valko}:  \emph{Online decision-making on graphs: Smoothness and Side Observations},
 Presented  at DaSciM, LIX, \'Ecole Polytechnique, France, April 14th, 2015 
({\sf X 2015})
\citembullet
{\bf  Michal Valko}:  \emph{Bandits on Graphs: Exploiting Smoothness and Side Observations},
 Presented  at CMLA, ENS Cachan, France, December 16th, 2014 
({\sf ENS 2014})
\citembullet
{\bf  Michal Valko}:  \emph{Optimistic Optimization},
 Presented  at MIST conference, Fa\v ckovsk\' e sedlo, Slovakia, January 7th, 2014
({\sf MIST 2014})
\citembullet
{\bf  Michal Valko}:  \emph{Sequential Face Recognition with Minimal Feedback},
Presented at 30 minutes of Science, Lille, May 2nd, 2013 ({\sf Inria 2013})
\citembullet
{\bf  Michal Valko}:  \emph{One Class Learning From Streams of Unlabeled Data},
Presented at Large-scale Online Learning and Decision Making Workshop,
April 28th, 2012 ({\sf LSOLDM 2012})
\citembullet
{\bf  Michal Valko}:  \emph{Scaling Graph-Based Algorithms}, Presented at
LAMPADA workshop, July 20th, 2012 ({\sf LAMPADA 2012})
\citembullet
{\bf  Michal Valko}:  \emph{Large Scale Sequential Learning}, opening speaker at
Slovak Oxford Science, April 28th, 2012 ({\sf Oxford UK 2012})
\citembullet
{\bf  Michal Valko}:  \emph{Adaptive Graph-Based Algorithms}, Presented on July
6th, 2011 at Microsoft Research Redmond ({\sf MSR Redmont 2011})
\citembullet
{\bf  Michal Valko}:  \emph{Online Semi-Supervised Learning}, Presented in 2011
at MPI T\"{u}bingen, Germany ({\sf MPI Tuebingen 2011})
\citembullet
{\bf  Michal Valko}:  \emph{Semi-supervised Learning with Random Walks on
Graphs}, Presented at 6th Comenius University Alumni conference ({\sf TAM 2009})
\end{category}
%\newpage

%% -------- Work experience --------------------------------------------
%
%\begin{category}{Work \\experience}
%\citem{Intel Labs Internship}, Intel, Santa Clara, CA (2010)\\
%Multi-manifold learning. Large scale semi-supervised learning.
%
%\citem{Intel Research Internship}, Intel, Santa Clara, CA (2009)\\
%Online semi-supervised learning. Max-margin structured prediction.
%
%\citem{Research Assistant}, University of Pittsburgh (2007 -- 2011)\\
%Conditional Anomaly Detection project: System for Anomaly Detection in Medicine
%
%\citem{Research Assistant}, University of Pittsburgh (2006)\\
%Bioinformatics: Tools for preprocessing, analysis of high-throughput proteomic and genomic data and biomarker discovery.
%
%\citem{Teaching Assistant}, University of Pittsburgh (Fall 2005)\\ 
%CS7 course: Introduction to Programming
%
%\citem{Research Assistant}, Institute of Normal and Pathological Physiology (2003 -- 2005)\\
%Slovak Academy of Sciences, Bratislava, Slovakia
%%\begin{itemize} 
%%\item 
%%\end{itemize}
%
%\citem{Research Fellow}, Centro de Intelig\^encia Artificial, (Spring 2005)\\
%Universidade Nova de Lisboa, Portugal
%
%
%\citem{Organizer and Lecturer}, Math Seminars in Slovakia (1998 -- 2005)\\
%Math Competitions, Math Summer Camps, Slovakia
%
%%\citem{Cashier/Cook/Pizza maker}, Work and Travel USA (Summer 2004) \\
%%Sandwich Haven \& Pizza, Marthas Vineyard, MA, USA
%%
%%\citem{Web developer/designer}, freelance (1996 -- 2004)\\ 
%%
%\end{category}



% -------- Professional Activities --------------------------------------------


\end{document}
